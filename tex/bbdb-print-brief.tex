%%% bbdb-print-brief.tex - for formatting address lists, one line per entry.

%%% Authors: Luigi Semenzato <luigi@paris.cs.berkeley.edu>
%%%          Boris Goldowsky <boris@cs.rochester.edu>
%%% Copyright (C) 1993 Boris Goldowsky
%%% Version: 3.91; 19Dec94

%%% For instructions on using this format file with BBDB, see bbdb-print.el
%%% which should have come bundled with this file; or write to
%%% boris@cs.rochester.edu.

%%% This file is part of the bbdb-print extensions to the Insidious
%%% Big Brother Database, which is for use with GNU Emacs.
%%%
%%% The Insidious Big Brother Database is free software; you can redistribute
%%% it and/or modify it under the terms of the GNU General Public License as
%%% published by the Free Software Foundation; either version 1, or (at your
%%% option) any later version.
%%%
%%% BBDB is distributed in the hope that it will be useful, but WITHOUT ANY
%%% WARRANTY; without even the implied warranty of MERCHANTABILITY or FITNESS
%%% FOR A PARTICULAR PURPOSE.  See the GNU General Public License for more
%%% details.
%%%
%%% You should have received a copy of the GNU General Public License
%%% along with GNU Emacs; see the file COPYING.  If not, write to
%%% the Free Software Foundation, 675 Mass Ave, Cambridge, MA 02139, USA.

%%% The address-list file should look something like this:

%%% \input file          % this format file's filename
%%% \setsize{6}          % point size of type to use
%%%                      % or \setpssize{6} to use PostScript fonts. (optional)
%%% \setseparator{3}     % which style of separators, 0-7
%%%
%%% \separator{A}        % include a separator here
%%%
%%% \beginrecord         % and start a record
%%% \name{A. Name}
%%% \phone{location: (xxx) xxx-xxxx}
%%% \address{1234 Main Street\\
%%% Anytown, XX 00000\\}
%%% \note{note name}{note text}
%%% \notes{blah blah}
%%% \endrecord
%%% 
%%% \endaddresses        % done
%%% \bye

\nopagenumbers
\raggedright
\tolerance=10000
\hbadness=10000
\parskip 0pt
\parindent=0pt  % was 10pt

%%%
%%% Fonts
%%%

\def\setsize#1{
  \font\rm=ecrm#1
  \font\bf=ecbx#1
  \font\it=\ifnum #1=6 ecti7 \else ecti#1 \fi
  \font\tt=ectt#1
  \font\bigbf=ecbx#1 scaled \magstep3
  \rm
  \baselineskip=#1pt
  \ifnum #1>9 \advance\baselineskip by 1pt \fi
}

\def\setpssize#1{
  \font\rm=ptmr at #1pt
  \font\bf=ptmb at #1pt
  \font\it=ptmri at #1pt
  \font\tt=pcrr at #1pt
  {\dimen0=#1pt\global\font\bigbf=ptmb at 1.8\dimen0}
  \rm
  \baselineskip=#1pt
}

%%%
%%% Define separator types
%%% 

\def\setseparator#1{
  \ifnum #1=1
    \def\sep##1{\line{\hrulefill}\smallskip\mark{##1}}
  \else \ifnum #1=2
    \def\sep##1{\hbox{\vrule\hskip -0.4pt\vbox{\hrule\smallskip
            \centerline{\bf{##1}}\smallskip\hrule}\hskip -0.4pt\vrule
            \mark{##1}}}
  \else \ifnum #1=3
    \def\sep##1{\hbox{\vrule\hskip -0.4pt\vbox{\hrule\smallskip
            \centerline{\bigbf{##1}}\smallskip\hrule}\hskip -0.4pt\vrule}
            \medskip\mark{##1}}
  \else \ifnum #1=4
    \def\sep##1{\smallskip\centerline{\bigbf{##1}}\medskip\mark{##1}}
  \else \ifnum #1=5
    \def\sep##1{\hrule\smallskip
            \centerline{\bigbf{##1}}\smallskip\hrule\medskip\mark{##1}}
  \else \ifnum #1=6
    \def\cute{$\sya\syb\syc\syd$}
    \def\revcute{$\syd\syc\syb\sya$}
    \let\sya=\heartsuit\let\syb=\spadesuit
      \let\syc\diamondsuit\let\syd=\clubsuit
    \def\cycle{\let\tmp=\sya\let\sya=\syb
                   \let\syb=\syc\let\syc=\syd\let\syd=\tmp}
    \def\sep##1{\smallskip
        \hbox to \hsize{\hfil\cute\hfil\bigbf{##1}\hfil\revcute\hfil}
        \cycle\medskip\mark{##1}}
  \else \ifnum #1=7
    \def\cute{$\sya\syb\syc\syd$}
    \def\revcute{$\syd\syc\syb\sya$}
    \let\sya=\heartsuit\let\syb=\spadesuit
    \let\syc=\diamondsuit\let\syd=\clubsuit
    \def\cycle{\let\tmp=\sya\let\sya=\syb\let\syb=\syc\let\syc=\syd
               \let\syd=\tmp}
    \def\sep##1{\hbox{\vrule\vbox{\hrule\smallskip
            \hbox to \hsize{\hfil\cute\hfil\bigbf{##1}\hfil\revcute\hfil}
            \smallskip\hrule}\vrule}\medskip\cycle\mark{##1}}
  \else
    \def\sep##1{\mark{##1}}
  \fi\fi\fi\fi\fi\fi\fi
  \def\separator##1{\noalign{\sep##1}}
} % end setseparator

%%%
%%% Macros for formatting the entries.
%%%

\def\\{}

\def\firstline#1#2{% the name and (maybe) the first phone number.
  {\bf #1}\hfil&\hfil#2&\hskip0pt}

\def\name#1{\firstline{#1}{}} % for backwards compatibility

\def\phone#1{\hfil#1&\hskip0pt}

\def\address#1{#1\hfil&\hskip0pt}

\def\comp#1{\hskip0pt}
\def\email#1{\hskip0pt}
\def\note#1#2{\hskip0pt}
\def\notes#1{\hskip0pt}

\def\beginrecord{\relax}
\def\endrecord{\cr}

\def\beginaddresses{\halign\bgroup&##\quad\cr}
\def\endaddresses{\egroup}

\def\today{\number\day\space
        \ifcase\month\or Jan\or Feb\or Mar\or Apr \or May\or June\or
        Jul\or Aug\or Sept\or Oct\or Nov\or Dec\fi
        \space\number\year}

%%% bbdb-print-brief.tex ends here.
